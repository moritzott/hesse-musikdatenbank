\documentclass[oneside, a4paper]{article}

\usepackage[utf8]{inputenc}
\usepackage[T1]{fontenc}

\usepackage[ngerman]{babel}
\usepackage{blindtext}

\author{Moritz Ott}
\title{Die Hermann-Hesse-Musikdatenbank}

\usepackage{cochineal}
\usepackage{sourcesanspro}
%\usepackage{epigrafica}
\usepackage{sectsty}

\allsectionsfont{\sffamily}
\usepackage{microtype}

\begin{document}

\maketitle
\thispagestyle{empty}
\newpage
\tableofcontents
\newpage
\begin{abstract}
Diese Anleitung enthält die grundlegenden Informationen zur Arbeit mit der Hermann-Hesse-Musikdatenbank. Dabei handelt es sich um eine Musikdatenbank mit Texten von Hermann Hesse über die Musik (eigene Aussagen und von fiktiven Erzählpersonen)
sowie einem Bedienprogramm zur komfortablen Arbeit mit dem Programm.
\end{abstract}
\newpage
\section{Einleitung}

\subsection{Beschreibung}
\blindtext

\subsection{Motivation}
\blindtext
\subsection{Technische Informationen}
\blindtext
\subsection{Ziele}
\blindtext
\subsection{Werkliste und Stand der Datenbank}
\blindtext\blindtext\blindtext
\section{Bedienung der Anwendung}
\blindtext\blindtext
\subsection{Installation}
\blindtext
\subsection{Programmstart}
\blindtext\blindtext
\subsection{Bedienung des Programms}
\blindtext
\subsection{Mit den Ergebnissen arbeiten}
\blindtext
\section{Erweiterungen}
\subsection{Programmerweiterungen}
\subsection{Die Datenbank selbst erweitern}
\blindtext

\end{document}